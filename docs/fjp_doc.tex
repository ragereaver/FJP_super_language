% Možné jazyky práce: czech, english
% Možné typy práce: BP (bakalářská), DP (diplomová)
\documentclass[czech]{thesiskiv}

%alternativa: 
%\declarationfemale

% Název práce
\title{Překladač jazyka do PL/0}

% Zařadit literaturu do obsahu
\usepackage[nottoc,notlot,notlof]{tocbibind}

% Umožňuje vkládání obrázků
\usepackage[pdftex]{graphicx}

\usepackage{amsmath}
\usepackage{array}
\usepackage{float}

\usepackage{ctable}
\usepackage{dsfont}
\usepackage{placeins}

% Odkazy v PDF jsou aktivní; navíc se automaticky vkládá
% balíček 'url', který umožňuje např. dělení slov
% uvnitř URL
\usepackage[pdftex]{hyperref}
\hypersetup{colorlinks=true,
  unicode=true,
  linkcolor=black,
  citecolor=black,
  urlcolor=black,
  bookmarksopen=true}

% Při používání citačního stylu csplainnatkiv
% (odvozen z csplainnat, http://repo.or.cz/w/csplainnat.git)
% lze snadno modifikovat vzhled citací v textu
\usepackage[numbers,sort&compress]{natbib}

\begin{document}
%
\maketitle
\thispagestyle{empty} 
\pagestyle{empty}
\tableofcontents
\addtocontents{toc}{\protect\thispagestyle{empty}}

\chapter{Zadání}
\label{sec:zadani}
\pagestyle{plain}
\setcounter{page}{1}
Cílem práce je vytvoření vlastního jazyka a překladače pro~tento jazyk.
Překládat jsme se rozhodli do~instrukční sady PL/0.
Při~vytváření jazyka jsme se snažili napodobit syntaxi jazyků java a C.
Od~vytvořeného jazyka jsme požadovali, aby uměl následující základní elementy:

\begin{itemize}
\item definice celočíselných proměnných
\item definice celočíselných konstant
\item přiřazení
\item základní aritmetiku a logiku (+, -, *, /, AND, OR, negace a závorky, operátory pro porovnání čísel)
\item cyklus (while)
\item jednoduchou podmínku (if bez else)
\item definice podprogramu (procedura, funkce, metoda) a jeho volání
\end{itemize}

\noindent Dále jsme se rozhodli jazyk rožšířit o~další složitější konstrukce.
Mezi složitější konstrukce, které jazyk umí patří:

\begin{itemize}
\item další typy cyklů (for, do...while, until, repeat...until)
\item else větev
\item datový typ boolean a logické operace s ním
\item rozvětvená podmínka (switch, case)
\item podmíněné přiřazení / ternární operátor (min = (a < b) ? a : b;)
\item pole a práce s jeho prvky
\item parametry předávané hodnotou
\item návratová hodnota podprogramu
\item komentáře
\end{itemize}

\noindent Za~plně funkční překladač pro~jazyk, které umí tyto konstrukce by mělo být uděleno 24 bodů.
Další bonusový bod by mohl být za~realizaci komentářů, které nebyly uvedeny v seznamu možných konstrukcí.

\chapter{Popis řešení}

\chapter{Syntaxe}
V této kapitole bude krátce okomentována a ukázána syntaxe jazyka.

\section{Definice proměnných a přiřazení}

Deklarace proměnných:

\texttt{int cislo;}

\texttt{
boolean logika;}

\noindent Deklarace konstant:

\texttt{const int CISLO = 5;}

\texttt{const boolean LOGIKA = true;}

\noindent Konstanty musí mít vždy přiřazenou hodnotu již při deklaraci.

\noindent Deklarace s přiřazením: 

\texttt{int cislo = 5;}

\texttt{boolean logika = true;}

\noindent Přiřazení:

\texttt{cislo = 5 + 3;}

\texttt{cislo = cislo + 1;}

\section{Podmínky}
\subsection{Podmínka if}
Část programu ve~větvi \texttt{if} se provede, pokud je splněná podmínka.
V~podmínce se může využívat všech logických operátorů, viz příklady.
Zároveň je možné doplnit na~konci větve \texttt{if} větev \texttt{else}, která 
se provede v~případě, že není splněná podmínka.

\noindent{Ukázka podmínky:}

\texttt{if(!(2 < 3 \&\& 1 > 0) || 1 != 0)} 

\texttt{\{...\}}

\texttt{else\{...\} }

\noindent Podmínka musí být v~závorkách následována ihned po~příkazu \texttt{if}.
Za~podmínkou ve~složených závorkách se pak nachází část kódu, který se má vykonat v~případě splněné podmínky.

\subsection{Switch}
V podmínce \texttt{switch} se musí nacházet pouze celé číslo nebo proměnná  \texttt{int}.
Podle dané hodnoty se provede určitý \texttt{case} uvnitř \texttt{switch}.
Zároveň lze na~konci \texttt{switch} udělat větev \texttt{default}, která 
se provede v~případě, že žádný \texttt{case} neodpovídal hodnotě v~podmínce.
Narozdíl od~jazyků C a java, se vždy provede pouze jeden \texttt{case}.

\noindent{Ukázka podmínky:}

\texttt{switch(2)\{}

\texttt{    case 1: }

\texttt{    case 2: int a = 2;}

\texttt{    default: int b = 0;\}}

\subsection{Ternární operátor}
Jazyk umožňuje i zkrácený zápis podmínky \texttt{if}, případně podmíněného přiřazení.

\noindent{Ukázka ternární podmínky}

\texttt{  (cislo < 2) ? cislo = 2 : cislo = 3;}

\noindent{Ukázka podmíněného přiřazení}

\texttt{  cislo = (cislo < 2) ? 2 : 3;}

               
\section{Cykly}
Cykly slouží k~určitému opakování stejného kódu.

\subsection{While}
Cyklus, který se provádí dokud je splněná podmínka.
Platí zde stejná pravidla jako v~podmínce \texttt{if}.

\noindent{Ukázka cyklu: }

\texttt{while(cislo < 3)\{}

\texttt{    cislo = cislo + 1;}

\texttt{\}}

\subsection{Do...while}
Podmínka se ověřuje až na~konci cyklu, tedy program se vykoná vždy alespoň jednou.

\noindent{Ukázka cyklu: }

\texttt{do\{}

\texttt{    cislo = cislo + 1;}

\texttt{\}while(cislo < 3);}

\subsection{Until}
Podobný cyklus jako \texttt{while}, akorát se provádí pokud podmínka je nesplněná.
Jakmile se podmínka splní, cyklus končí.

\noindent{Ukázka cyklu: }

\texttt{until(cislo > 3)\{}

\texttt{    cislo = cislo + 1;}

\texttt{\}}

\subsection{Repeat...until}
Podobný cyklus jako \texttt{until}, akorát podmínka se ověřuje až na~konci cyklu.
Program se tedy vykoná alespoň jedenkrát.

\subsection{For}
Cyklus s~určitým počtem opakování. Podmínka se skládá ze tří částí.
V~první části musí být deklarace proměnné s~přiřazením počáteční hodnoty.
V~druhé části musí být podmínka, při~její splnění se bude cyklus provádět.
V~poslední části je pak operace, která se provede na~konci cyklu.

\noindent{Ukázka cyklu: }

\texttt{for(int i = 0; i < 3; i = i + 1)\{} 

\texttt{    ...}

\texttt{\}}
               
\section{Pole}


\section{Funkce}
Program lze členit do~podprogramů pomocí funkcí.
Funkce musí být definovány na~začátku programu, při~definici je důležité klíčové slovo
\texttt{function}. Funkcím lze předávat parametry a zároveň funkce můžeš vracet jednu hodnotu, viz příklad.

\noindent{Ukázka funkce: }

\texttt{int function soucet(int a, int b)\{}

\texttt{    return a + b;}

\texttt{\}}

\noindent{Volání funkce: }

\texttt{int c = soucet(1, 2);}

\section{Komentáře}
Komentáře slouží k~označení části kódu, která se nebude překládat do~instrukcí.
Realizovány byly blokové komentáře, které jsou označeny sekvencí \texttt{/*} na~začátku bloku
a \texttt{*/} na~konci bloku.

\noindent{Ukázka komentářů: }

\texttt{/* tohle je komentar */}

               
\chapter{Testovací příklady}
Testování funkčnosti řešení bylo realizováno pomocí testovacích souborů, pro~které jsme měli správné posloupnosti instrukcí.
Při~změnách v~překladači se pak pouštěl překlad těchto testovacích souborů a provnávali se výstupní instrukce se~správnými.

\noindent Testovací soubory lze najít ve~složce \textit{tests/testFiles}, programy napsané v~našem jazyce mají příponu \textit{.sll}.
Přeložené programy do~instrukční sady PL/0 mají příponu \textit{.pl}.

\noindent Některé kratší ukázky a výstupní instrukce přiložím zde.

\section{Test přiřazení}

Program:

\texttt{int a = 5;}

\texttt{int mn, ob = 5 + a, i = 3, or;}

\texttt{boolean c = true; }

\texttt{a = 3;}

\texttt{const int TEST = 4;}

\texttt{int b = TEST;}

\texttt{int d = b;}

\texttt{int e = b;}

\texttt{if (a < 5) \{ }

\texttt{    b = 3; }
    
\texttt{\} else \{ }

\texttt{    b = 8; }
    
\texttt{\} }

\texttt{c = a == b;}



\noindent{Instrukce:}

\texttt{0	JMP	0	1}

\texttt{1	INT	0	13}

\texttt{2	LIT	0	5 }

\texttt{3	STO	0	3 }

\texttt{4	LIT	0	0 }

\texttt{5	STO	0	4 }

\texttt{6	LIT	0	5 }

\texttt{7	LOD	0	3 }

\texttt{8	OPR	0	2 }
                   
\texttt{9	STO	0	5  }

\texttt{10	STO	0	6 }

\texttt{11	LIT	0	0  }

\texttt{12	STO	0	7   }

\texttt{13	LIT	0	1    }

\texttt{14	STO	0	8     }

\texttt{15	LIT	0	3      }

\texttt{16	STO	0	3       }

\texttt{17	LIT	0	4      }

\texttt{18	STO	0	9 }

\texttt{19	LOD	0	9  }

\texttt{20	STO	0	10  }

\texttt{21	LOD	0	10   }

\texttt{22	STO	0	11    }

\texttt{23	LOD	0	10     }

\texttt{24	STO	0	12      }

\texttt{25	LOD	0	3        }

\texttt{26	LIT	0	5 }

\texttt{27	OPR	0	10 }

\texttt{28	JMC	0	32  }

\texttt{29	LIT	0	3    }

\texttt{30	STO	0	10    }

\texttt{31	JMP	0	34     }

\texttt{32	LIT	0	8       }

\texttt{33	STO	0	10       }

\texttt{34	LOD	0	3         }

\texttt{35	LOD	0	10         }

\texttt{36	OPR	0	8           }

\texttt{37	STO	0	8            }

\texttt{38	RET	0	0             }



\section{Test cyklu a podmínek}

Zde je otestovaný pouze cyklus \texttt{for} a podmínka \texttt{if}.
Všechny cykly jsou testovány v~souboru \textit{tests/testFiles/cykly/testCycles.sll}.

\noindent Program:

\texttt{int i;}

\texttt{for(int k = 0; k < 3; k = k + 1)\{}
    
\texttt{    if(i < 3)\{ }

\texttt{        i = i - 1;}

\texttt{    \}             }

\texttt{    else\{          }

\texttt{        i = i * k;  }

\texttt{    \}            }

\texttt{\}                 }

\noindent Instrukce:

\texttt{0	JMP	0	1}

\texttt{1	INT	0	5}

\texttt{2	LIT	0	0 }

\texttt{3	STO	0	3  }

\texttt{4	LIT	0	0   }

\texttt{5	STO	0	4    }

\texttt{6	LOD	0	4     }

\texttt{7	LIT	0	3      }

\texttt{8	OPR	0	10      }

\texttt{9	JMC	0	28       }

\texttt{10	LOD	0	3       }

\texttt{11	LIT	0	3        }

\texttt{12	OPR	0	10        }

\texttt{13	JMC	0	19         }

\texttt{14	LOD	0	3           }

\texttt{15	LIT	0	1 }

\texttt{16	OPR	0	3  }

\texttt{17	STO	0	3   }

\texttt{18	JMP	0	23   }

\texttt{19	LOD	0	3     }

\texttt{20	LOD	0	4      }

\texttt{21	OPR	0	4       }

\texttt{22	STO	0	3        }

\texttt{23	LOD	0	4         }

\texttt{24	LIT	0	1          }

\texttt{25	OPR	0	2           }

\texttt{26	STO	0	4            }

\texttt{27	JMP	0	6             }

\texttt{28	RET	0	0              }


\section{Testování polí}

\section{Testování funkcí}
Ukázka funkce pro~součet dvou čísel.

\noindent Program:

\texttt{int function soucet (int a, int b) \{}
 
\texttt{    return a + b;}

\texttt{\} }

\texttt{int c = soucet(1, 2); }

\noindent Instrukce: 

\texttt{0	JMP	0	1              }

\texttt{1	INT	0	8              }

\texttt{2	LIT	0	0             }

\texttt{3	STO	0	3            }

\texttt{4	LIT	0	0           }

\texttt{5	STO	0	4          }

\texttt{6	LIT	0	1         }

\texttt{7	STO	0	4        }

\texttt{8	LIT	0	2       }

\texttt{9	STO	0	5      }

\texttt{10	CAL	0	14  }

\texttt{11	LOD	0	3  }

\texttt{12	STO	0	6 }

\texttt{13	RET	0	0         }

\texttt{14	INT	0	5        }

\texttt{15	LOD	1	4       }

\texttt{16	STO	0	3      }

\texttt{17	LOD	1	5     }

\texttt{18	STO	0	4    }

\texttt{19	LOD	0	3   }

\texttt{20	LOD	0	4  }

\texttt{21	OPR	0	2 }

\texttt{22	STO	1	3}

\texttt{23	RET	0	0}



\chapter{Závěr}
Semestrální práci se podařilo úspěšně dokončit. Nicméně během tvorby jazyka a překladače
jsme narazili na řadu problémů a obtíží. Nejhorší byl začátek, kdy bylo potřeba sestavit gramatiku jazyka
a zprovoznit nástroj ANTLR pro~parsování programu. Dále se pak naučit jak funguje PL/0, a co znamenají jednotlivé instrukce.

Práce byla poměrně rozsáhlá a bylo potřeba mnoho úsilí, aby byly splněny minimální požadavky na~rozsah.
Na~druhou stranu byla práce originální a poskytla nám pohled do~fungování překladačů,
jak se z~nám známeho programovacího jazyka stane posloupnost strojových instrukcí použitelných pro~procesor.

Vzhledem k~tomu, že práce byla dělána ve~dvojicích, se bylo vždy možno poradit, když jsme si nevěděli rady.
Zároveň si myslíme, že i komunikace byla lepší, než kdyby jsme tuto práci dělali v~početnějším týmu.
Ačkoliv si nedovedeme představit, kolik času by bylo potřeba nad~touto~prací strávit, abychom dosáhli maximálního počtu bodů.

Celý projekt byl veden na~githubu na~adrese \href{https://github.com/tuslm/FJP\_{super}\_{language}}{FJP\_{super}\_{language}}\footnote{Adresa v případě nefunkčnosti odkazu: https://github.com/tuslm/FJP\_{super}\_{language}}.
Za~semestrální práci, tak jak bylo řečeno v~kapitole \ref{sec:zadani}, by jsme očekávali 24 bodů. Případně
bonusový bod za~implementaci komentářů.

\end{document}

